

\subsection*{Che\-M\-P\-S2\-: a spin-\/adapted implementation of D\-M\-R\-G for ab initio quantum chemistry}

Copyright (C) 2013 Sebastian Wouters \href{mailto:sebastianwouters@gmail.com}{\tt sebastianwouters@gmail.\-com}

This program is free software; you can redistribute it and/or modify it under the terms of the G\-N\-U General Public License as published by the Free Software Foundation; either version 2 of the License, or (at your option) any later version.

This program is distributed in the hope that it will be useful, but W\-I\-T\-H\-O\-U\-T A\-N\-Y W\-A\-R\-R\-A\-N\-T\-Y; without even the implied warranty of M\-E\-R\-C\-H\-A\-N\-T\-A\-B\-I\-L\-I\-T\-Y or F\-I\-T\-N\-E\-S\-S F\-O\-R A P\-A\-R\-T\-I\-C\-U\-L\-A\-R P\-U\-R\-P\-O\-S\-E. See the G\-N\-U General Public License for more details.

You should have received a copy of the G\-N\-U General Public License along with this program; if not, write to the Free Software Foundation, Inc., 51 Franklin Street, Fifth Floor, Boston, M\-A 02110-\/1301 U\-S\-A.

\subsubsection*{How to acknowledge Che\-M\-P\-S2}

To acknowledge Che\-M\-P\-S2, please cite


\begin{DoxyEnumerate}
\item S. Wouters, W. Poelmans, P.\-W. Ayers and D. Van Neck, Che\-M\-P\-S2\-: a free open-\/source spin-\/adapted implementation of the density matrix renormalization group for ab initio quantum chemistry, Submitted to Computer Physics Communications, \href{http://arxiv.org/abs/1312.2415}{\tt ar\-Xiv\-:1312.\-2415} \begin{DoxyVerb} @article{CheMPS2cite1,
     author = {Sebastian Wouters and Ward Poelmans and Paul W. Ayers and Dimitri {Van Neck}},
     title = {CheMPS2: a free open-source spin-adapted implementation of the density matrix renormalization group for ab initio quantum chemistry},
     journal = {ArXiv e-prints},
     archivePrefix = "arXiv",
     eprint = {1312.2415},
     primaryClass = "cond-mat.str-el",
     year = {2013}
 }
\end{DoxyVerb}

\end{DoxyEnumerate}


\begin{DoxyEnumerate}
\item S. Wouters, P.\-A. Limacher, D. Van Neck and P.\-W. Ayers, Longitudinal static optical properties of hydrogen chains\-: Finite field extrapolations of matrix product state calculations, The Journal of Chemical Physics 136, 134110 (2012), \href{http://dx.doi.org/10.1063/1.3700087}{\tt doi\-:10.\-1063/1.3700087} \begin{DoxyVerb} @article{CheMPS2cite2,
     author = {Sebastian Wouters and Peter A. Limacher and Dimitri {Van Neck} and Paul W. Ayers},
     title = {Longitudinal static optical properties of hydrogen chains: Finite field extrapolations of matrix product state calculations},
     journal = {The Journal of Chemical Physics},
     year = {2012},
     volume = {136},
     number = {13}, 
     pages = {134110},
     doi = {10.1063/1.3700087} 
 }
\end{DoxyVerb}

\end{DoxyEnumerate}

\subsubsection*{List of files in the Che\-M\-P\-S2 library}

\begin{DoxyVerb}CheMPS2/CASSCF.cpp
CheMPS2/CASSCFdebug.cpp
CheMPS2/CASSCFhamiltonianrotation.cpp
CheMPS2/CASSCFnewtonraphson.cpp
CheMPS2/ConvergenceScheme.cpp
CheMPS2/DMRG.cpp
CheMPS2/DMRGmpsio.cpp
CheMPS2/DMRGoperators.cpp
CheMPS2/DMRGtechnics.cpp
CheMPS2/FourIndex.cpp
CheMPS2/Hamiltonian.cpp
CheMPS2/Heff.cpp
CheMPS2/HeffDiagonal.cpp
CheMPS2/HeffDiagrams1.cpp
CheMPS2/HeffDiagrams2.cpp
CheMPS2/HeffDiagrams3.cpp
CheMPS2/HeffDiagrams4.cpp
CheMPS2/HeffDiagrams5.cpp
CheMPS2/Irreps.cpp
CheMPS2/PrintLicense.cpp
CheMPS2/Problem.cpp
CheMPS2/Sobject.cpp
CheMPS2/SyBookkeeper.cpp
CheMPS2/TensorA.cpp
CheMPS2/TensorB.cpp
CheMPS2/TensorC.cpp
CheMPS2/TensorD.cpp
CheMPS2/TensorDiag.cpp
CheMPS2/TensorF0Cbase.cpp
CheMPS2/TensorF0.cpp
CheMPS2/TensorF1.cpp
CheMPS2/TensorF1Dbase.cpp
CheMPS2/TensorL.cpp
CheMPS2/TensorO.cpp
CheMPS2/TensorQ.cpp
CheMPS2/TensorS0Abase.cpp
CheMPS2/TensorS0.cpp
CheMPS2/TensorS1Bbase.cpp
CheMPS2/TensorS1.cpp
CheMPS2/TensorSwap.cpp
CheMPS2/TensorT.cpp
CheMPS2/TensorX.cpp
CheMPS2/TwoDM.cpp
CheMPS2/TwoIndex.cpp
CheMPS2/include/CASSCF.h
CheMPS2/include/ConvergenceScheme.h
CheMPS2/include/DMRG.h
CheMPS2/include/FourIndex.h
CheMPS2/include/Gsl.h
CheMPS2/include/Hamiltonian.h
CheMPS2/include/Heff.h
CheMPS2/include/Irreps.h
CheMPS2/include/Lapack.h
CheMPS2/include/Options.h
CheMPS2/include/Problem.h
CheMPS2/include/Sobject.h
CheMPS2/include/SyBookkeeper.h
CheMPS2/include/TensorA.h
CheMPS2/include/TensorB.h
CheMPS2/include/TensorC.h
CheMPS2/include/TensorD.h
CheMPS2/include/TensorDiag.h
CheMPS2/include/TensorF0Cbase.h
CheMPS2/include/TensorF0.h
CheMPS2/include/TensorF1Dbase.h
CheMPS2/include/TensorF1.h
CheMPS2/include/Tensor.h
CheMPS2/include/TensorL.h
CheMPS2/include/TensorO.h
CheMPS2/include/TensorQ.h
CheMPS2/include/TensorS0Abase.h
CheMPS2/include/TensorS0.h
CheMPS2/include/TensorS1Bbase.h
CheMPS2/include/TensorS1.h
CheMPS2/include/TensorSwap.h
CheMPS2/include/TensorT.h
CheMPS2/include/TensorX.h
CheMPS2/include/TwoDM.h
CheMPS2/include/TwoIndex.h
\end{DoxyVerb}


Please note that these files are documented with Doxygen-\/readable comments. Search for the section \char`\"{}\-Build\char`\"{} in R\-E\-A\-D\-M\-E to see how a manual can be generated from these comments.

\subsubsection*{List of files to perform test runs}

\begin{DoxyVerb}tests/test1.cpp
tests/test2.cpp
tests/test3.cpp
tests/test4.cpp
tests/test5.cpp
tests/test6.cpp
tests/matrixelements/CH4_N10_S0_c2v_I0.dat
tests/matrixelements/H6_N6_S0_d2h_I0.dat
tests/matrixelements/N2_N14_S0_d2h_I0.dat
tests/matrixelements/O2_CCPVDZ.dat
\end{DoxyVerb}


These test files illustrate how to use the Che\-M\-P\-S2 library. They only require a very limited amount of memory (order 10-\/100 M\-B).

\subsubsection*{Matrix elements from Psi4}

Che\-M\-P\-S2 has a Hamiltonian object which is able to read in matrix elements from a plugin to Psi4 \href{http://www.psicode.org}{\tt Psi4, Ab initio quantum chemistry}, which works on version psi4.\-0b5 and higher.

To make use of this feature, build Psi4 with the plugin option, and then run\-: \begin{DoxyVerb}> psi4 --new-plugin mointegrals
> cd mointegrals
\end{DoxyVerb}


Now, replace the file {\ttfamily mointegrals.\-cc} with either\-:


\begin{DoxyEnumerate}
\item {\ttfamily ./mointegrals/mointegrals.cc\-\_\-\-P\-R\-I\-N\-T} to print the matrix elements as text. Examples of output generated with this plugin can be found in {\ttfamily ./tests/matrixelements}
\end{DoxyEnumerate}


\begin{DoxyEnumerate}
\item {\ttfamily ./mointegrals/mointegrals.cc\-\_\-\-S\-A\-V\-E\-H\-A\-M} to store all unique matrix elements (remember that there is eightfold permutation symmetry) in binary form with H\-D\-F5. See the Doxygen manual for more information on \hyperlink{classCheMPS2_1_1Hamiltonian}{Che\-M\-P\-S2\-::\-Hamiltonian}.
\end{DoxyEnumerate}

For case 2, the {\ttfamily Makefile} should be adjusted. Replace {\ttfamily } with {\ttfamily  -\/\-L\$\{Che\-M\-P\-S2\-\_\-\-B\-I\-N\-A\-R\-Y\-\_\-\-D\-I\-R\}/\-Che\-M\-P\-S2/ -\/l\-Che\-M\-P\-S2} and replace {\ttfamily } with {\ttfamily  -\/\-I\$\{Che\-M\-P\-S2\-\_\-\-S\-O\-U\-R\-C\-E\-\_\-\-D\-I\-R\}/\-Che\-M\-P\-S2/include/}

To compile the Psi4 plugin, run\-: \begin{DoxyVerb}> make
\end{DoxyVerb}


\subsubsection*{Build}

\paragraph*{1. Build Che\-M\-P\-S2 with C\-Make}

Che\-M\-P\-S2 can be build with C\-Make. The files \begin{DoxyVerb}./CMakeLists.txt
./CheMPS2/CMakeLists.txt
./tests/CMakeLists.txt
\end{DoxyVerb}


provide a minimal compilation. Start in {\ttfamily ./} and run\-: \begin{DoxyVerb}> mkdir build
> cd build
\end{DoxyVerb}


C\-Make generates makefiles based on the user's specifications\-: \begin{DoxyVerb}> CXX=option1 cmake .. -DMKL=option2 -DBUILD_DOCUMENTATION=option3
\end{DoxyVerb}


Option1 is the c++ compiler; typically {\ttfamily g++} or {\ttfamily icpc} on Linux. Option2 can be {\ttfamily O\-N} or {\ttfamily O\-F\-F} and is used to switch on the intel math kernel library. Option3 can be {\ttfamily O\-N} or {\ttfamily O\-F\-F} and is used to switch on doxygen documentation.

To compile, run\-: \begin{DoxyVerb}> make
\end{DoxyVerb}


\paragraph*{2. Testing Che\-M\-P\-S2}

To test Che\-M\-P\-S2, start in {\ttfamily ./build}, and run\-: \begin{DoxyVerb}> cd tests/
> ./test1
> ./test2
> ./test3
> ./test4
> ./test5
> ./test6
\end{DoxyVerb}


The tests should end with a line stating whether or not they succeeded. They only require a very limited amount of memory (order 10-\/100 M\-B).

\paragraph*{3. Doxygen documentation}

To build and view the Doxygen manual, the documentation flag should have been on\-: {\ttfamily -\/\-D\-B\-U\-I\-L\-D\-\_\-\-D\-O\-C\-U\-M\-E\-N\-T\-A\-T\-I\-O\-N=O\-N}. Start in {\ttfamily ./build} and run\-: \begin{DoxyVerb}> make doc
> cd LaTeX-documents
> make
> evince refman.pdf &
> cd ../html
> firefox index.html &
\end{DoxyVerb}


\subsubsection*{User manual}

Doxygen output can be generated, see the section \char`\"{}\-Build\char`\"{} in R\-E\-A\-D\-M\-E.

For a more concise overview of the concepts and ideas used in Che\-M\-P\-S2, please read the code release paper "Che\-M\-P\-S2\-: a free open-\/source spin-\/adapted implementation of the density matrix renormalization group for ab initio quantum chemistry", one of the references in C\-I\-T\-A\-T\-I\-O\-N\-S. 